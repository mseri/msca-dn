\section{Excellence \DNtag{REL-EVA-RE}}
\note{Should start on page 5; aim for 10 to 12 pages.}
\note{
    Excellence - aspects to be taken into account.
    \begin{compactitem}
        \item Quality and pertinence of the project’s research and innovation objectives (and the extent towhich they are ambitious, and go beyond the state of the art).
        \item Soundness of the proposed methodology (including interdisciplinary approaches, consideration of the gender dimension and other diversity aspects if relevant for the research project, and the quality and appropriateness of open science practices).
        \item Quality and credibility of the training programme (including transferable skills, inter/multidisciplinary, inter-sectoral and gender as well as other diversity aspects).
        \item Quality of the supervision (including mandatory joint supervision for industrial and joint doctorate projects).
    \end{compactitem}
}

\subsection{Quality and pertinence of the project's research and innovation objectives}
\note{and the extent to which they are ambitious, and go beyond the state of the art. The action should be divided in \textbf{Work packages}  and described in Table 3.1a under the Implementation section.}

\subsubsection{Introduction, objectives and overview of the research programme}
\note{It should be explained how the individual projects of the recruited researchers will be integrated into --- and contribute to --- the overall research programme. All proposals should describe the research projects in the context of a doctoral training programme.  Are the objectives measurable and verifiable? Are they realistically achievable?}

\subsubsection{Pertinence and innovative aspects of the research programme}
\note{in light of the current state of the art and existing programmes / networks / doctoral research trainings.}
\note{Describe how your project goes beyond the state-of-the-art, and the extent the proposed work is ambitious.}

\subsection{Soundness of the proposed methodology}
\note{including interdisciplinary approaches, consideration of the gender dimension and other diversity aspects if relevant for the research project, and the quality and appropriateness of open science practices. Includes interdusciplinary approaches, consideration of the gender dimension and other diversity aspects if relevant for the research project, and the quality and appropriateness of open science practices.}

\subsubsection{Overall methodology:}
\note{Describe and explain the overall methodology including the concepts,
models and assumptions that underpin your work. Explain how this will enable you to deliver your project’s objectives. Refer to any important challenges you may have identified in the chosen methodology and how you intend to overcome them.}

\subsubsection{Integration of methods and disciplines to pursue the objectives:}
\note{Explain how expertise and methods from different disciplines will be brought together and integrated in pursuit of your objectives. If you consider that an inter-disciplinary approach is unnecessary in the context of the proposed work, please provide a justification.}

\subsubsection{Gender dimension and other diversity aspects:}
\note{Describe how the gender dimension and other diversity aspects are taken into account in the project’s research and innovation content. If you do not consider such a gender dimension to be relevant in your project, please provide a justification.}

\note{This point is about the gender dimension of the planned research and innovation, not the gender balance of the teams carrying out the project.}

\subsubsection{Open science practices:}
\note{Describe how appropriate open science practices are implemented as
an integral part of the proposed methodology. Show how the choice of practices and their implementation are adapted to the nature of your work, in a way that will increase the chances of the project delivering on its objectives. If you believe that none of these practices are appropriate for your project, please provide a justification here.}

\note{Open science is an approach based on open cooperative work and systematic sharing of knowledge and tools as early and widely as possible in the process. Open science practices include early and open sharing of research (for example through preregistration, registered reports, pre-prints, or crowd-sourcing); research output management; measures to ensure reproducibility of research outputs; providing open access to research outputs (such as publications, data, software, models, algorithms, and workflows); participation in open peer-review; and involving all relevant knowledge actors including citizens, civil society and end users in the co-creation of R\&I agendas and contents (such as citizen science). Please note that this question does not refer to outreach actions that may be planned as part of communication, dissemination and exploitation activities. These aspects should instead be described below under ‘Impact’.}

\subsubsection{Research data management and management of other research outputs:}
\note{Applicants generating/collecting data and/or other research outputs (except for publications) during the project must provide maximum 1 page on how the data will be managed in line with the FAIR principles (Findable, Accessible, Interoperable, Reusable), addressing the following (the description should be specific to your project):
\begin{compactitem}
    \item Types of data/research outputs/research outputs (e.g. experimental, observational, images, text, numerical) and their estimated size; if applicable, combination with, and provenance of, existing data.
    \item Findability of data/research outputs: Types of persistent and unique identifiers (e.g. digital object identifiers) and trusted repositories that will be used.
    \item Accessibility of data/research outputs: IPR considerations and timeline for open access (if open access not provided, explain why); provisions for access to restricted data for verification purposes.
    \item Interoperability of data/research outputs: Standards, formats and vocabularies for data and metadata.
    \item Reusability of data/research outputs: Licenses for data sharing and re-use (e.g. Creative Commons, Open Data Commons); availability of tools/software/models for data generation and validation/interpretation /re-use.
    \item Curation and storage/preservation costs; person/team responsible for data management and quality assurance.
\end{compactitem}
Proposals selected for funding under Horizon Europe will need to develop a detailed data management plan (DMP) for making their data findable, accessible, interoperable and reusable (FAIR) as a deliverable at mid-term and revised towards the end of a project's lifetime. Such a fully detailed DMP is not requested at the proposal stage, however the sub-heading ``Research data management and management of other research outputs'' is mandatory at the proposal stage.
}

\subsection{Artificial intelligence (if applicable to the proposal):}
\note{
If the activities proposed involve the use and/or development of AI-based systems and/or techniques, applicants must provide explanations on the technical robustness of the proposed system(s).
If you plan to use, develop and/or deploy artificial intelligence (AI) based systems and/or techniques you must demonstrate their technical robustness. AI-based systems or techniques should be, or be developed to become:
\begin{compactitem}
    \item	technically robust, accurate and reproducible, and able to deal with and inform about possible failures, inaccuracies and errors, proportionate to the assessed risk they pose 
    \item	socially robust, in that they duly consider the context and environment in which they operate 
    \item	reliable and function as intended, minimizing unintentional and unexpected harm, preventing unacceptable harm and safeguarding the physical and mental integrity of humans
    \item	able to provide a suitable explanation of their decision-making processes, whenever they can have a significant impact on people’s lives.
\end{compactitem}  
}

\subsection{Quality and credibility of the training programme}
\note{including transferable skills, inter/multi-disciplinary, inter-sectoral and gender as well as other aspects}

\subsubsection{Overview and content structure of the doctoral training programme}
\note{including network-wide training events and complementarity with those programmes offered locally at the participating organisations (please include table 1).}

\msccaption{Table 1: Main Network-Wide Training Events, Conferences and Contribution of Beneficiaries}
\begin{msctable}{|>{\ra}p{10mm}|>{\ra}p{85mm}|>{\ra}p{20mm}|>{\ra}p{20mm}|>{\ra}p{20mm}|}
    \colorrow
    \hline &
    \textbf{Main Training Events \& Conferences} &
    \textbf{ECTS (if any)} &
    \textbf{Lead Institution} &
    \textbf{Action Month (estimated)} \\
    \hline
    &&&& \\
    \hline
\end{msctable}

\subsection{Quality of the supervision}
\note{including mandatory joint supervision for industrial and joint doctorate projects}

\subsubsection{Qualifications and supervision experience of supervisors}

\subsubsection{Quality of supervision arrangements for DN}

\subsubsection{Quality of the mandatory joint supervision arrangements}
\note{For DN-ID and DN-JD}

\note{To avoid duplication, the role and scientific profile of the supervisors should only be listed in the "Participating Organisations" tables (see section 8 below).
The following section of the European Charter for Researchers refers specifically to supervision:
\textbf{Supervision}
Employers and/or funders should ensure that a person is clearly identified to whom researchers can refer for the performance of their professional duties, and should inform the researchers accordingly.
Such arrangements should clearly define that the proposed supervisors are sufficiently expert in supervising research, have the time, knowledge, experience, expertise and commitment to be able to offer the research doctoral candidate appropriate support and provide for the necessary progress and review procedures, as well as the necessary feedback mechanisms.}

\note{Supervision is one of the crucial elements of successful research. Guiding, supporting, directing, advising and mentoring are key factors for a researcher to pursue his/her career path. In this context, all MSCA-funded projects are encouraged to follow the recommendations outlined in the Guidelines for MSCA supervision.}

\DNtage{REL-EVA-RE}

%%% Local Variables:
%%% mode: latex
%%% TeX-master: "master"
%%% TeX-PDF-mode: t
%%% End:
