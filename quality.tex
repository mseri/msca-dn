\section{Quality and Efficiency of the Implementation \DNtag{QUA-LIT-QL} \DNtag{WRK-PLA-WP}  \DNtag{CON-SOR-CS} \DNtag{PRJ-MGT-PM}}
\note{\textbf{Quality and efficiency of the implementation – aspects to be taken into account}
\begin{compactitem}
  \item Quality and effectiveness of the work plan, assessment of risks and appropriateness of the effort assigned to work packages.
  \item Quality, capacity and role of each participant, including hosting arrangements and extent to which the consortium as a whole brings together the necessary expertise.
\end{compactitem}}

\subsection{Quality and effectiveness of the work plan, assessment of risks and appropriateness of the effort assigned to work packages}

\subsubsection{Work Packages (WP) list}
\note{please include table 3.1a}

\msccaption{Table 3.1a: Work Package (WP) List}
\begin{msctable}{|>{\ra}p{10mm}|>{\ra}p{35mm}|>{\ra}p{15mm}|>{\ra}p{10mm}|>{\ra}p{10mm}|>{\ra}p{20mm}|>{\ra}p{20mm}|>{\ra}p{25mm}|}
    \colorrow
    \hline
    \textbf{WP No.} &
    \textbf{WP Title} &
    \textbf{Lead Beneficiary No.} &
    \textbf{Lead Beneficiary Short Name} &
    \textbf{Start Month} &
    \textbf{End Month} &
    \textbf{Activity Type} &
    \textbf{Researcher Involvement} \\
    \hline
    &&&&&&& \\
    \hline
\end{msctable}

\note{The WP names are defined in common.tex}

\note{Activity Type: e.g. research, management, dissemination, etc.}

\note{The Work Packages should reflect the research objectives. Only brief headings and overviews of the Work Packages should be presented in Table 3.1a. More details in terms of actual implementation should be provided in Table 3.1b.}

\subsubsection{Description of Work Packages}
\note{please include table 3.1b}

\msccaption{Table 3.1b: Description of Work Packages}
\begin{mscwp}{MX -- MY}{Title (e.g. including Research, Training,  Management, Communication and Dissemination, etc.)}{Lead}
    \mscwppar{Objectives}{%
    }\\
    \hline
    \mscwppar{Description of Work and Role of Specific Beneficiaries / Associated partners}{%
    }\\
    \mscwptask{Task name}{Task description}\\
    \hline
    \mscwppar{Description of Deliverables}{%
    }\\
\end{mscwp}

\note{Objectives are possibly broken down into tasks, indicating lead participant and role of other participating organisations. For
each task, quantify the amount of work. Provide enough detail to justify the resources requested and clarify why the work is needed and who will do it.
Deliverables linked to each WP are listed in Table 3.1c (no need to repeat the information here).
}

\subsubsection{Deliverables list}
\note{please include table 3.1c}

\note{%
    Type of deliverable:
    \begin{compactdesc}
    \item[R] Report
    \item[ADM] Administrative (website completion, recruitment completion, etc.)
    \item[PDE] dissemination and/or exploitation of results
    \item[OTHER] Other, including coordination.
    \end{compactdesc}
    Dissemination level:
    \begin{compactdesc}
    \item[PU] Public: fully open, e.g. web
    \item[SEN] Sensitive: restricted to consortium, other designated entities (as appropriate) and Commission services; Please consider that deliverables marked as ``PU'' will automatically be published on CORDIS once approved: the applicants should therefore consider the relevance of marking a deliverable as ``PU''
    \item[CI] Classified: classified information as intended in Commission Decision 2001/844/EC
    \end{compactdesc}
}
\msccaption{Table 3.1c: Deliverables List}
\begin{msctable}{|>{\ra}p{15mm}|>{\ra}p{50mm}|>{\ra}p{15mm}|>{\ra}p{20mm}|>{\ra}p{15mm}|>{\ra}p{15mm}|>{\ra}p{15mm}|}
    \hline
    \colorrow
    \multicolumn{7}{|l|}{\textbf{Scientific Deliverables}} \\
    \hline
    \colorrow
    \textbf{Number} &
    \textbf{Deliverable Title} &
    \textbf{WP No.} &
    \textbf{Lead Beneficiary Short Name} &
    \textbf{Type} &
    \textbf{Dissem. Level} &
    \textbf{Due Date (in months)} \\
    \hline
    % Split it here if need be
\end{msctable}
\begin{msctable}{|>{\ra}p{15mm}|>{\ra}p{50mm}|>{\ra}p{15mm}|>{\ra}p{20mm}|>{\ra}p{15mm}|>{\ra}p{15mm}|>{\ra}p{15mm}|}
    \hline
    \colorrow
    \multicolumn{7}{|l|}{\textbf{Management, Training, Recruitment and Dissemination Deliverables}} \\
    \hline
    \colorrow
    \textbf{Deliverable Number} &
    \textbf{Deliverable Title} &
    \textbf{WP No.} &
    \textbf{Lead Beneficiary Short Name} &
    \textbf{Type} &
    \textbf{Dissem. Level} &
    \textbf{Due Date} \\
    \hline
\end{msctable}

\note{The deliverable numbers are ordered by delivery date and numbered WPNo.DeliverableNoWithinWP}
\note{The deliverables should be divided into scientific deliverables and management, training, recruitment and dissemination deliverables. Scientific deliverables have technical/scientific content specific to the action. The number of deliverables in a given Work Package must be reasonable and commensurate with the Work Package content. Note that during implementation, the submission of these deliverables to the REA will be a contractual obligation.
Note that, if the proposal is successful, several mandatory deliverables will be added during the Grant Agreement preparation such as the establishment of a supervisory board of the network, due at month 2; the progress report, due at month 13; the career development plan etc. (full list in the MSCA Work Programme – Definitions section, paragraph 1.6).\\
Due date: The schedule should indicate the number of months elapsed from the start of the action (Month 1).
}

\subsubsection{Milestones list}
\note{please include table 3.1d}

\msccaption{Table 3.1d: Milestones List}
\begin{msctable}{|>{\ra}p{15mm}|>{\ra}p{50mm}|>{\ra}p{15mm}|>{\ra}p{20mm}|>{\ra}p{15mm}|>{\ra}p{35mm}|}
    \hline
    \colorrow
    \textbf{Number} &
    \textbf{Title} &
    \textbf{Related Work Packages} &
    \textbf{Lead Beneficiary} &
    \textbf{Due Date} &
    \textbf{Means of Verification} \\
    \hline
\end{msctable}

\note{Means of Verification: Show how the consortium will confirm that the milestone has been attained. Refer to indicators if appropriate. For example: a laboratory prototype completed and running flawlessly; software released and validated by a user group; field survey complete and data quality validated.}

\subsubsection{Recruitment table per beneficiary}
\note{please include table 3.1e}

\msccaption{Table 3.1e: Recruitment Table per Beneficiary}
\begin{msctable}{|>{\ra}p{30mm}|>{\ra}p{30mm}|>{\ra}p{30mm}|>{\ra}p{30mm}|>{\ra}p{30mm}|}
    \colorrow
    \hline
    \textbf{Researcher No.} &
    \textbf{Recruiting Participant (short name)} &
    \textbf{PhD awarding entities} &
    \textbf{Planned Start Month 0--45} &
    \textbf{Duration (months) 3--36 (48 for DN-JD)} \\
    \hline
    &&&& \\
    \hline
    Total & & & & XXX \\
    \hline
\end{msctable}

\note{If a Doctoral Candidate is recruited by more than one beneficiary, please indicate this in the table accordingly. The total must be included in the table as its last row.}

\subsubsection{Individual research projects, including secondment plan}
\note{please include table 3.1f}

\msccaption{Table 3.1f: Individual Research Projects}

\note{%
\begin{mscrp}
    X & X & X & X & X & X \\
    \hline
    \mscrppar{Project Title and Work Package(s) to which it is related}{
    }\\ \hline
    \mscrppar{Objectives}{%
    }\\ \hline
    \mscrppar{Expected results}{%
    }\\ \hline
    \mscrppar{Planned secondment(s)}{ Host, superviso, timing, length and purpose %
    }\\ \hline
    \mscrppar{Enrolment in Doctoral degree(s)}{%
    DN-JD specific: institutions where the researcher will be enrolled to obtain a joint/double or multiple doctoral degree should be included.
    DN and DN-ID: institution where the researcher will be enrolled to obtain a doctoral degree should be included
    }\\
\end{mscrp}
}

\note{If applicable and relevant, linkages between the individual research projects and the work packages should be summarised here (one table/fellow).}

\note{Fellow: e.g researcher 1\\Start Date and Duration: always as months (from start of consortium in the first case)\\Deliverables: Refer to table 3.1b}

\subsubsection{Progress monitoring and evaluation of individual research projects}

\subsubsection{Implementation risks}
\note{please include table 3.1g}

\msccaption{Table 3.1e: Implementation Risks \DNtag{RSK-MGT-RM}}
\begin{msctable}{|>{\ra}p{60mm}|>{\ra}p{15mm}|>{\ra}p{90mm}|}
    \hline
    \colorrow
    \textbf{Description of risk} (likelihood / severity) &
    \textbf{WPs involved} &
    \textbf{Proposed risk-mitigation measures} \\
    \hline
\end{msctable}

\note{Please list the critical managerial, scientific and technical risks, relating to project implementation and detail the risk mitigation measures. Please include dealing with scientific misconduct as one of the critical risks for research. Please also refer to any important challenges you may have identified in the chosen methodology and how you intend to overcome them.}

\note{Description of risk. Indicate level of (i) likelihood, and (ii) severity.
A critical risk is a plausible event or issue that could have a high adverse impact on the ability of the project to achieve its objectives.\\
Level of likelihood to occur: Low/medium/high\\
The likelihood is the estimated probability that the risk will materialise even after taking account of the mitigating measures put in place.\\
Level of severity: Low/medium/high\\
The relative seriousness of the risk and the significance of its effect.
}

\DNtage{RSK-MGT-RM}

\subsubsection{Joint admission, selection, supervision, monitoring and assessment procedures}
\note{For DN-JD, if not applicable please remove.}

\DNtage{CON-SOR-CS} \DNtage{PRJ-MGT-PM}

\subsection{Quality, capacity and role of each participant, including hosting arrangements and extent to which the consortium as a whole brings together the necessary expertise}

\subsubsection{Appropriateness of the infrastructure and capacity of each participating organisation}
\note{as outlined in Section 8 (Participating Organisations), in light of the tasks allocated to them in the action}

\subsubsection{Consortium composition and exploitation of participating organisations' complementarities}
\note{explain the compatibility and coherence between the tasks attributed to each beneficiary/associated partner in the action, including in light of their experience; Show how this includes expertise in social sciences and humanities, open science practices, and gender aspects of R\&I, as appropriate.}

\subsubsection{Commitment of beneficiaries and associated partners to the programme}
\note{please see also section 8. The role of associated partners and their active contribution to the research and training activities should be explained and justified.}

\subsubsection{Funding of non-associated third countries (if applicable)}
\note{Only entities from EU Member States, from Horizon Europe Associated Countries or from countries listed in the HE Programme guide are automatically eligible for EU funding. If one or more of the beneficiaries requesting EU funding is based in a country that is not automatically eligible for such funding, the application shall explain in terms of the objectives of the action why such funding would be essential. Only in exceptional cases will these organisations receive EU funding. The same applies for international organisations other than IERO.}

\DNtage{QUA-LIT-QL} \DNtage{WRK-PLA-WP}

%%% Local Variables:
%%% mode: latex
%%% TeX-master: "master"
%%% TeX-PDF-mode: t
%%% End:
