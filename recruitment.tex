\section{Recruitment Strategy}

\note{including how the project will strive to adhere to the Code of Conduct for the recruitment of researcher.}

\note{
The following sections of the European Code of Conduct for the Recruitment of Researchers refer specifically to recruitment and selection:\\
\textbf{Recruitment}
Employers and/or funders should establish recruitment procedures which are open, efficient, transparent, supportive and internationally comparable, as well as tailored to the type of positions advertised. Advertisements should give a broad description of knowledge and competencies required, and should not be so specialised as to discourage suitable applicants. Employers should include a description of the working conditions and entitlements, including career development prospects. Moreover, the time allowed between the advertisement of the vacancy or the call for applications and the deadline for reply should be realistic.\\
\textbf{Selection}
Selection committees should bring together diverse expertise and competences and should have an adequate gender balance and, where appropriate and feasible, include members from different sectors
(academic and non-academic) and disciplines, including from other countries and with relevant experience to assess the candidate. Whenever possible, a wide range of selection practices should be used, such as external expert assessment and face-to-face interviews. Members of selection panels should be adequately trained.}
