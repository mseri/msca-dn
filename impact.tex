\section{Impact \DNtag{IMP-ACT-IA}}

\note{
    \textbf{Impact – aspects to be taken into account}
    \begin{compactitem}
        \item Contribution to structuring doctoral training at the European level and to strengthening European innovation capacity, including the potential for: a) meaningful contribution of the non-academic sector to the doctoral training, as appropriate to the implementation mode and research field b) developing sustainable elements of doctoral programmes.
        \item Credibility of the measures to enhance the career perspectives and employability of researchers and contribution to their skills development.
        \item Suitability and quality of the measures to maximise expected outcomes and impacts, as set out in the dissemination and exploitation plan, including communication activities.
        \item The magnitude and importance of the project’s contribution to the expected scientific, societal and economic impacts.
    \end{compactitem}
}

\subsection{Contribution to structuring doctoral training at the European level and to strengthening European innovation capacity}
\note{Including the potential for: a) meaningful contribution of the non-academic sector to the doctoral training, as appropriate to the implementation mode and research field, this could include (non exhaustively) e.g. meaningful exposure of Doctoral Candidates to the non-academic sector through secondments, contribution of the non-academic sector to the research and training.
b) developing sustainable elements of doctoral programmes after the end of the DN funding.This could include, for example sustainability of training programmes and transferable skills training offered at local or network-wide level (e.g. training programmes open to doctoral students outside the consortium, training courses still available and running after the end of the project); sustainable cooperation/long lasting collaboration and secondment opportunities (e.g. how will the consortium partners continuing to publish together, and complement their research work and exchange research visit and doctoral students after the end of the project).}

\subsection{Credibility of the measures to enhance the career perspectives and employability of researchers and contribution to their skills development}
\note{In this section, please explain the impact of the research and training on the fellows' careers prospects. Explain how the project and the training will give technical and transferable skills to the fellows, which will improve their employability in academia and/or the industry.}

\subsection{Suitability and quality of the measures to maximise expected outcomes and impacts, as set out in the dissemination and exploitation plan, including communication activities}

\note{Concrete plans for sections 2.3 must be included in the corresponding table 3.1 b Description of Work Packages.}

\note{Note that the following sections of the European Charter for Researchers refer specifically to public engagement and dissemination:\\
\DNtag{COM-DIS-VIS-CDV}\\
\textbf{Dissemination, Exploitation of Results}
All researchers should ensure, in compliance with their contractual arrangements, that the results of their research are disseminated and exploited, e.g. communicated, transferred into other research settings or, if appropriate, commercialised. Senior researchers, in particular, are expected to take a lead in ensuring that research is fruitful and that results are either exploited commercially or made accessible to the public (or both) whenever the opportunity arises.\\
\textbf{Public Engagement}
Researchers should ensure that their research activities are made known to society at large in such a way that they can be understood by non-specialists, thereby improving the public's understanding of
science. Direct engagement with the public will help researchers to better understand public interest in priorities for science and technology and also the public's concerns.\\
\DNtage{COM-DIS-VIS-CDV}}

\subsubsection{Plan for the dissemination and exploitation activities, including communication activities:}
\note{Describe the planned measures to maximise the impact of your project by providing a first version of your ‘plan for the dissemination and exploitation including communication activities’. Describe the dissemination, exploitation and communication measures that are planned, the target group(s) addressed (e.g. scientific community, end users, financial actors, public at large), with objectives, and how these activities and the fulfilment of these objectives will be monitored, and using which indicators.}

\note{Regarding communication measures and public engagement strategy, the aim is to inform and reach out to society and show the activities performed, and the use and the benefits the project will have for citizens. Activities must be strategically planned, with clear objectives, start at the outset and continue through the lifetime of the project. The description of the communication activities needs to state the main messages as well as the tools and channels that will be used to reach out to each of the chosen target groups.
In case your proposal is selected for funding, a more detailed plan will need to be provided as a mandatory project deliverable submitted at mid-term stage with an update towards the end of the project.}

\subsubsection{Strategy for the management of intellectual property, foreseen protection measures}
\note{such as patents, design rights, copyright, trade secrets, etc., and how these would be used to support exploitation.}

\note{If your project is selected, you will need an appropriate consortium agreement to manage (amongst other things) the ownership and access to key knowledge (IPR, research data etc.). Where relevant, these will allow you, collectively and individually, to pursue market opportunities arising from the project. Please note that although a detailed IP management plan is not expected at this stage, an outline of the strategy for the management of IP is mandatory at the proposal stage.
All measures should be proportionate to the scale of the project, and should contain concrete actions to be implemented both during and after the end of the project, e.g. standardisation activities. Your plan should give due consideration to the possible follow-up of your project, once it is finished. In the justification, explain why each measure chosen is best suited to reach the target group addressed. Where relevant, describe the measures for a plausible path to commercialise the innovations.
If exploitation is expected primarily in non-associated third countries, justify by explaining how that exploitation is still in the Union's interest.
}


\subsection{The magnitude and importance of the project’s contribution to the expected scientific societal and economic impacts (project’s pathways towards impact)}
\note{Provide a narrative explaining how the project’s results are expected to make a difference in terms of impact, beyond the immediate scope and duration of the project. The narrative should include the components below, tailored to your project. Please justify and explain how the stated impacts are credible, relevant, and achievable.}

\note{Be specific, referring to the effects of your project, and not R\&I in general in this field. State the target groups that would benefit.
Only include such outcomes and impacts where your project would make a significant and direct contribution. Avoid describing very tenuous links to wider impacts.
Give an indication of the magnitude and importance of the project’s contribution to the expected outcomes and impacts, should the project be successful. Provide quantified estimates where possible and meaningful. `Magnitude'' refers to how widespread the outcomes and impacts are likely to be. For example, in terms of the size of the target group, or the proportion of that group, that should benefit over time; `Importance' refers to the value of those benefits. For example, number of additional healthy life years; efficiency savings in energy supply.
}

\subsubsection{Expected scientific impact}
\note{e.g. contributing to specific scientific advances, across and
within disciplines, creating new knowledge, reinforcing scientific equipment and instruments, computing systems (i.e. research infrastructures)}

\subsubsection{Expected economic/technological impact}
\note{e.g. bringing new products, services, business processes to the market, increasing efficiency, decreasing costs, increasing profits, contributing to standards’ setting, etc.}

\subsubsection{Expected societal impact}
\note{decreasing CO2 emissions, decreasing avoidable mortality, improving policies and decision-making, raising consumer awareness}

\DNtage{IMP-ACT-IA}

%%% Local Variables:
%%% mode: latex
%%% TeX-master: "master"
%%% TeX-PDF-mode: t
%%% End:
